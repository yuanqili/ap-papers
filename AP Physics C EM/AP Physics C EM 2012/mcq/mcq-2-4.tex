\textbf{See the instruction for questions \inteval{\value{question}+1} to \inteval{\value{question}+3}.}

\begin{figure}[H]
    \center
    \includegraphics[scale=0.25]{images/img-002-004.png}
\end{figure}

Three particles having charges of $+q$, $+Q$, and $-Q$ are placed at the corners of an equilateral triangle of side $a$, as shown above.

% Multiple Choice Question 2
\begin{questions}\setcounter{question}{1}\question
The net force on the particle with charge $+q$ due to the other two charges is in the plane of the page and directed

\begin{choices}
\choice vertically upward
\choice vertically downward
\choice horizontally to the right
\choice horizontally to the left
\choice toward the charge $-Q$
\end{choices}\end{questions}

% Multiple Choice Question 3
\begin{questions}\setcounter{question}{2}\question
The magnitude of the force on the particle with charge $+q$ due to the other two charges is

\begin{oneparchoices}
\choice $\dfrac{k q Q}{a}$
\choice $\dfrac{2 k q Q}{a}$
\choice $\dfrac{2 k q Q}{a^{2}}$
\choice $\dfrac{2 k q Q \sin 60^{\circ}}{a^{2}}$
\choice $\dfrac{2 k q Q \cos 60^{\circ}}{a^{2}}$
\end{oneparchoices}\end{questions}

% Multiple Choice Question 4
\begin{questions}\setcounter{question}{3}\question
The potential energy of the particle with charge $+q$ due to the other two charges is

\begin{oneparchoices}
\choice zero
\choice $\dfrac{-2 k Q}{a} $
\choice $\dfrac{k q Q}{a} $
\choice $\dfrac{2 k q Q}{a}$
\choice $\dfrac{2 k q Q}{a^{2}}$
\end{oneparchoices}\end{questions}

